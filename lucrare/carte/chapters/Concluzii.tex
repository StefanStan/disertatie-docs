Am creat o aplicație pornind de la necesitatea prezentată în capitolul \textit{Introducere}. Scopul acesteia este de a oferi o metodă facila de a restaura o baza de date la un moment de timp oferit de către utilizator. Acesta nu necesită cunoștințe de baze de date, sau de devOps (nu e nevoie să interacționeze cu scripturi de UNIX).
\par
Pentru a realiza cele menționate anterior am studiat modalitatea bazei de date PostgreSQL de management intern a transacțiilor și am utilizat-o întro manieră ce îmi permite restaurarea bazei de date la un moment în timp cu o granularitate fină, la secundă.
\par
De asemenea, am studiat modalitatea sistemului de operare UNIX ce permite creare de partiții virtuale ce pot fi oricând modificate în ceea ce privește mărimea fără a închide baza de date. Am combinat aceasta funcționaliate a sistemului de operare împreuna cu soluția propusă, pentru a realiza cele prezentate cu un timp minim de inaccesibilitate a bazei de date. \textit{Capitolul 3} prezinta în detaliu aceste aspecte.
\par
Am propus o arhitectură a sistemului de fișiere bazat pe studiul prezentat anterior și am utilizat-o în implementarea propriu-zisa. Cât despre aceasta, am folosit serviciile Amazon Cloud (EC2, EBS) pentru a face deploy soluției (scripturi ce interacționează cu baza de date, aplicație backend și aplicație frontend).
\par
In opinia mea, prin studiul realizat și prin intermediul soluției propuse am reușit să realizez ce mi-am propus și am oferit o modalitate de management a bazelor de date și de restaurare a acestora la starea existentă un moment de timp. Prin urmare am realizat o aplicație usor de utilizat de catre persoane ce nu au cunoștințe de baze de date, sistem de operare UNIX.