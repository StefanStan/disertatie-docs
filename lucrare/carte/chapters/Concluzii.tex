Am creat o aplicatie pornind de la necesitatea prezentata in capitolul \textit{Introducere}. Scopul acesteia este de a oferi o metoda facila de a restaura o baza de date la un moment de timp oferit de catre utilizator. Acesta nu necesita cunostinte de baze de date, sau de devOps (nu e nevoie sa interactioneze cu scripturi de UNIX).
\par
Pentru a realiza cele mentionate anterior am studiat modalitatea bazei de date PostgreSQL de management intern a transactiilor si am utilizat-o intro maniera ce imi permite restaurarea bazei de date la un moment in timp cu o granularitate fina, la secunda.
\par
De asemenea, am studiat modalitatea sistemului de operare UNIX ce permite creare de partitii virtuale ce pot fi oricand modificate in ceea ce priveste marimea fara a inchide baza de date. Am combinat aceasta functionaliate a sistemului de operare impreuna cu solutia propusa, pentru a realiza cele prezentate cu un timp minim de inaccesibilitate a bazei de date. \textit{Capitolul 3} prezinta in detaliu aceste aspecte.
\par
Am propus o arhitectura a sistemului de fisiere bazat pe studiul prezentat anterior si am utilizat-o in implementarea propriu-zisa. Cat despre aceasta, am folosit serviciile Amazon Cloud (EC2, EBS) pentru a face deploy solutiei (scripturi ce interactioneaza cu baza de date, aplicatie backend si aplicatie frontend).
\par
In opinia mea, prin studiul realizat si prin intermediul solutiei propuse am reusit sa realizez ce mi-am propus si am oferit o modalitate de management a bazelor de date si de restaurare a acestora la starea existenta un moment de timp. Prin urmare am realizat o aplicatie usor de utilizat de catre persoane ce nu au cunostinta de baze de date, sistem de operare UNIX.