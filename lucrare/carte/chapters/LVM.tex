În zilele de început ale serverelor Linux se făcea managementul spațiului de stocare prin crearea de partiții pe disc. Chiar dacă această abordare funcționează, ea prezintă anumite dezavantaje, iar cea mai importantă este aceeaa că discurile sunt inflexibile. Acesta este motivul pentru care a fost introdus Logical Volume Manager. Chiar dacă nu este posibil să schimbăm în mod dinamic marimea unei partiții care ramane fără spațiu, acest lucru este rezolvat lucrând cu LVM. LVM oferă multe avantaje ce vor fi prezentate în cele ce urmează.
\section{Arhitectura LVM}
În arhitectura LVM se pot distinge mai multe layere. Primul nivel lucrează direct cu componentele de stocare. Acesta este creat într-un mod abstract și poate folosi orice medii de stocare.
\par
Mediile de stocare trebuie marcate ca "physical volumes", ceea ce le face să poată fi folosite de către LVM. O componentă de stocare marcată ca "physical volume" poate fi adăugata unui "volume group", layer abstract peste toate mediile de stocare. Abstractizarea constă în faptul că acesta nu este ceva fix, poate fi redimensionat când este necesar, ceea ce face posibil să se adauge mai mult spațiu la nivelul "volume group" atunci când "logical volumes" ce fac parte din el rămân făra spațiu. Concret, dacă se rămâne fără spațiu pe un logical volume se ia din spațiul liber de la nivelul volume group-ului din care face parte. În cazul în care nici la acel nivel nu mai există spațiu liber se adaugă un disc nou în volume group. 
\par
Intuitiv, deasupra nivelului de volume groups se află logical volumes. Acestea nu actionează ca medii de stocare directe, ele își iau spațiul necesar din discurile aflate în volume group. Astfel că, un logical volume poate avea spațiul de stocare distribuit pe mai multe physical volumes (medii de stocare).
\section{Funcționalități LVM}
Există multe motive pentru care LVM este foarte util. Cel mai important motiv este acela că oferă flexibilitate pentru managementul spațiului de stocare. Partițiile nu mai sunt limitate de către spațiul de stocare al discurilor. Deasemenea, este posibil să se reducă mărimea unui logical volume, dar doar daca sistemul de fisiere ce a fost creat suporta redimensionare. De exemplu, sistemul de fisiere Ext4 suporta redimensionare in timp ce XFX, care este sistemul de fisiere implicit de pe RHEL (Red Hat Enterprise Linux) nu suporta.
\par
Un alt motiv pentru care administratorii de sistem folosesc LVM este posibilitatea de a crea salvari. O salvare tine minte starea curenta a logical volume si poate fi folosita pentru il a restaura daca este nevoie.
\par
Al treilea avantaj important este acela ca avem posibilitatea de a schimba usor elementele hardware stricate. Daca un hard disk se strica, datele pot fi mutate in cadrul volume group prin comanda pvmove si un nou harddisk se poate adauga dinamic fara a avea sistemul cazut.
\section{Crearea Logical Volumes}
Crearea Logical Volumes presupune crearea a trei layere in arhitectura LVM. In primul rand trebuie create PV(physical volumes), apoi urmeaza sa se creeze VG(volume group) la care se adauga physical volumes create anterior. In ultimul rand, trebuie create LV(logical volumes) pe care le vom folosi ca si "partitii" efective.
\par
Inainte de a folosi utilitarele LVM trebuie sa marcam partitiile hardware ca fiind de tipul LVM. Daca se foloseste utilitara fdisk impreuna cu un disc de tip MBR, atunci partitia trebuie pusa de tipul 8e. Altfel, daca se foloseste gdisk impreuna cu un disc de tip GUID, atunci partitia este de tip 8300.
\subsection{Pasi pentru a crea un Physical Volume}
Avem nevoie de un harddisk care are spatiu liber (nepartitionat). Pentru scopul explicarii se va folosi un disk /dev/vdb pentru a crea partitia. Fiecare disc are un nume diferit astfel ca pasii vor trebui ajustati pentru fiecare disc.
\begin{itemize}
\item Se deschide un shell si se executa \textbf{fdisk /dev/vdb}
\item Se apasa \textbf{n} pentru a crea partitia. Se selecteaza \textbf{p} pentru a o marca partitie primara si se accepta numerotarea implicita.
\item Se apasa \textbf{Enter} de doua ori si astfel se creeza o partitie folosind tot spatiul de pe discul \textbf{/dev/vdb}
\item Fiind intors la prompter-ul fdisk, se apasa t pentru a schimba tipul partitiei create. Fiind o singura partitie ce foloseste intreg spatiul aceasta va fi selectata implicit, altfel ar putea intreba care pentru care partitie sa se schimbe tipul. 
\item Utilitara fdisk intreaba ce tip va avea partitia aleaa. Scrie 8e. Apasa w pentru a scrie schimbarile efectuate in acesti pasi. Daca se primeste un mesaj conform caruia tabela de partitii nu a putut fi actualizata trebuie dat un restart. 
\end{itemize}
Acum ca partitia a fost creata trebuie marcata ca LVM physical volume. Pentru asta se va executa \textbf{pvcreate /dev/vdb1}. Ar trebui primit urmatorul mesaj: Physical volume “/dev/vbd1” successfully created. Pentru a verifica existenta physical volume nou creat putem executa in shell comanda pvs ce listeaza physical volume existente.
\subsection{Pasi pentru a crea un Volume Group}
Acum ca am creat un physical volume trebuie sa il asignam unui volume group. Este posibil sa adauga physical volumes la un volume group existent (a se citi manualul din linux pentru comanda \textbf{vgextend}), insa acum un volume group nou si vom adauga un physical volume la el. Aceasta este procedura relativ usoara, o singura comanda, vgcreate, ce primeste numele noului volume group precum si physical volume ce va fi adaugat initial. De exemplu: \textbf{vgcreate vgdata /dev/vdb1}.
\par Este complet la alegerea utilizatorului ce nume alege pentru volume groups. In acest exemplu am ales sa incep numele cu vg, ceea ce le face usor de gasit daca vre-odata avem foarte multe. 
\par Dupa ce s-a create volume group putem executa comanda \textbf{vgs} pentru a lista volume groups existente.
\subsection{Pasi pentru a crea Logical Volumes}
Acum ca s-a creat un volume group putem incepe sa cream logical volumes. Procedura este un pic mai complicata decat crearea physical volumes sau volume groups, deoarece sunt mai multe alegeri ce trebuie facute. Cand se creeaza un logical volumes trebuie specificate numele si marimea.
\par
Marimea logical volume poate fi specificata prin valoare absoluta folosind optiunea \textbf{-L}. De exemplu folosind \textbf{-L 5G} se creeaza un logical volume cu marimea de 5GB. O alternativa este sa folosim optiunea \textbf{- l} pentru a specifica o marime relativa. De exemplu, se foloseste \textbf{- l 50\% FREE} pentru a folosi jumatate din spatiul liber la nivelul volume group. Urmatorul argument este numele volume group la care va fi asignat logical volume nou creat si din care se va lua spatiu necesar. Un parametru optional este \textbf{- n} ce specifica numele noului logical volume. De exemplu, comanda \textbf{lvcreate -n lvvol1 -L 100M vgdata} creeaza un logical volume cu numele lvvol folosind 100MB spatiu din volume group cu numele vgdata (creat anterior).
\subsection{Modificarea marimii unui Logical Volume}
Motivul folosirii logical volumes in locul utilizarii directe a unor discuri este acela ca la nevoie putem foarte usor sa adaugam spatiu intr-un logical volume, in timp ce daca de exemplu avem de adaugat in cazul discurilor trebuie sa adaugam un alt disc mai mare si sa copiem totul din discul initial, moment in care sistemul trebuie sa fie oprit. Daca de exemplu avem un folder /tenant1 in care tinem datele din baza de date pentru un client. Sa ne imaginal ca acesta trimite la un disc fizic de 100GB. Pentru a putea sa oferim mai mult spatiu in acest director va fi nevoie sa oprim baza de date, sa mutam toate datele pe un alt disc de 200GB de ex si apoi sa repornim sistemul. In cazul in care /tenant1 este un director la care a fost montat un logical volume putem folosi urmatoare procedura pentru a mari spatiul fara a opri sistemulȘ
\begin{itemize}
\item Verificam ca in volume group din care face parte logical volume ce va fi extins se afla spatiu liber. Daca nu exista spatiu liber adauga un nou physical volume.
\item Executam comanda \textbf{vgextend} pentru a extinde spatiul din logical volume.
\end{itemize}
Putem utiliza comanda \textbf{lvs} ce listeaza date despre logical volumes existente pentru a vedea spatiul liber al logical volume. Pentru detalii despre comanda \textbf{vgextend} a se executa comanda \textbf{man vgextend} ce afiseaza manualul din Linux.
\newpage
\section{Explicatii despre conventia de numire a LVM}
Din acest moment putem incepe sa folosim logical volume nou creat. Pentru a face asta trebuie sa stim cum sa ne adresam acestuia. Entitatile LVM pot fi adresate in moduri multiple. Cea mai simpla metoda este sa adresam un logical volume precum: /dev/vgname/lvname. Astfel ca, daca am creat un logical volume cu numele lvdata, ce isi ia spatiu dintrun volume group cu numele vgdata, calea entitatii va fi /dev/vgdata/lvdata.
\par
Device mapper poarta un rol important in denumirea LVM logical volumes. Device mapper (abreviat dm) este o interfata generica folosita de kernelul Linux pentru a adresa device de stocare. Device mapper este folosit de catre multiple sisteme: LVM volumes, dar si de software RAID etc. Aceste device sunt create doua locatii: device numerotate in secventa plasate in directorul /dev, precum /dev/dm-0, /dev/dm-1 si asa mai departe. Deoarece aceste denumiri nu ofera nicio informatie despre deviceuri ci confuzeaza, se folosesc scurtaturi create in directorul /dev/mapper. Acestea folosesc sablonul vgname-lvname pentru nume. Astfel ca, daca avem device /dev/vgdata/lvdata putem sa ne adresam folosind si /dev/mapper/vgdata-lvdata. Cand lucram cum LVM logical volumes putem folosi oricare din aceste tipuri de adresare.