În crearea aplicației am văzut o necesitate exitentă întâlnită la dezvoltatorii de software și am încercat să o rezolv. Procesul general de restabilire a unei baze de date este descris în \textit{Capitolul 1}.
\par
În \textit{Capitolul 2} se prezintă într-o manieră detaliată o metodă de organizare a sistemului de fișiere UNIX, urmând ca în \textit{Capitolul 3} să fie discutată arhitectura aleasă pentru dispunearea clusterului de baze de date pe sistemul de fișiere, precum și avantajele aduse de aceasta.
\par
Soluția software este trecută în revistă în \textit{Capitolul 4}. Astfel că, se prezintă arhitectura aplicației, urmată de fiecare componentă ce intră în alcătuirea ei. Tot în acest capitol este descrisă și setarea aplicației și a mediului de lucru în Amazon Web Services.
\par
\textit{Capitolul 6} oferă un ghid detaliat al aplicației, urmând ca în \textit{Capitolul 7} să fie discutate direcțiile de dezvoltare viitoare și idei de îmbunătățire ale platformei dezvoltate.