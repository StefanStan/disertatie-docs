\large
Avand în vedere contextul unui dezvoltator software, am observat în timpul experienței faptul ca pentru echipa efectivă de dezvoltare era destul de dificil de întreținut baza de date (incluzând deasemenea și operațiile de pornire/oprire, mai ales pentru juniori acest lucru putea pune probleme dacă baza de date se afla instalată pe o platforma UNIX). În cazul în care se dorea aplicarea unei strategii de point in time recovery pentru restabilirea bazei de date la o stare existentă la un moment dat, era nevoie de intervenția echipei de devOps, care se ocupa în mod expres de acest tip de operații, având cunoștințe avansate despre subiect.
\par
Astfel că, am observat necesitatea existenței unei aplicații care să ușureze operațiile asupra bazei de date, începand cu pornirea/oprirea/verificarea starii, până la întreaga operație de backup și point in time recovery(întoarcere la o stare existentă la un anumit moment dat). Această aplicație trebuia să fie intuitivă, ușor de folosit chiar și pentru cineva care nu avea cunoștințe de devOps, nefiind necesar accesul efectiv pe mașina de UNIX unde se afla clusterul de baze de date. Este de la sine înteles că aplicația ar trebui să detecteze toate bazele de date din clusterul de pe mașina pe care este încarcată și ar trebui să permită operațiile prezentate mai sus pentru fiecare instanță în parte.
\par
Impactele unei astfel de aplicații sunt atat de natură tehnică, le oferă dezvoltatorilor mai mult control asupra sistemului necesitând mai puține cunoștințe, precum și de business, nu mai trebuie angajată o echipă dedicată de devOps și prin urmare se reduc costurile.